%%%%%%%%%%%%%%%%%%%%%%%%%%%%%%%%%%%%%%%%%%%%%%%%%%%%%%%%%%%%%%%%%%%
%%                                                               %%
%% This is the sample.tex file for the jphilart document class.  %%
%%                                                               %%
%% You need the jphilart.cls in your current directory or in any %%
%% directory scanned for cls files by your latex engine.         %%
%%                                                               %%
%% Manual inclusion of page layout commands is useless.          %%
%%                                                               %%
%% Note that any complex file will produce delayed publication!  %%
%%                                                               %%
%%%%%%%%%%%%%%%%%%%%%%%%%%%%%%%%%%%%%%%%%%%%%%%%%%%%%%%%%%%%%%%%%%%

% for lead article use class option: leadarticle
\documentclass{jphilart}
\hypersetup{
    colorlinks,
    linkcolor={blue},
    citecolor={blue},
    urlcolor={blue}
}

%%%%%%%%%%%%%%%%%%%%%%%%%%%%%%%%%%%%%%%%%%%%%%%%%%%%%%%%%%%%%%%%%%%
%% The following items will be set by the Managing Editor.       %%
%%%%%%%%%%%%%%%%%%%%%%%%%%%%%%%%%%%%%%%%%%%%%%%%%%%%%%%%%%%%%%%%%%%

\pubyear{2018}
\pubmonth{0}
\volume{0}
\issue{0}
\firstpage{1}
\lastpage{1}
% \copyrightholder{}

%%%%%%%%%%%%%%%%%%%%%%%%%%%%%%%%%%%%%%%%%%%%%%%%%%%%%%%%%%%%%%%%%%%
%% Please add here your own packages (be minimalistic please!):  %%
%% Please avoid using exotic packages and keep things simple.    %%
%% It is not necessary to include ams* and graphicx packages     %%
%% since they are automatically included by the jphilart class.  %%
%%%%%%%%%%%%%%%%%%%%%%%%%%%%%%%%%%%%%%%%%%%%%%%%%%%%%%%%%%%%%%%%%%%
\usepackage{verbatim}
%\usepackage{cleveref}  % uncomment to use this package

%%%%%%%%%%%%%%%%%%%%%%%%%%%%%%%%%%%%%%%%%%%%%%%%%%%%%%%%%%%%%%%%%%%
%% Please add your own macros and environments below             %%
%% If possible, avoid using \def and use instead \newcommand     %%
%%%%%%%%%%%%%%%%%%%%%%%%%%%%%%%%%%%%%%%%%%%%%%%%%%%%%%%%%%%%%%%%%%%
\startlocaldefs
\newcommand{\ABS}[1]{\left(#1\right)} % example of author macro
\newcommand{\jphilart}{\emph{jphilart}\ } % another example of author macro
\newtheorem{theorem}{Theorem} % example of author environment
\newtheorem*{fact*}{Fact} % another example of author environment
\newtheorem{proposition}[theorem]{Proposition} % another example of author environment
\endlocaldefs
%%%%%%%%%%%%%%%%%%%%%%%%%%%%%%%%%%%%%%%%%%%%%%%%%%%%%%%%%%%%%%%%%%%
%% No macro definitions below this line please!                  %%
%%%%%%%%%%%%%%%%%%%%%%%%%%%%%%%%%%%%%%%%%%%%%%%%%%%%%%%%%%%%%%%%%%%

\begin{document}
\begin{frontmatter}
%%%%%%%%%%%%%%%%%%%%%%%%%%%%%%%%%%%%%%%%%%%%%%%%%%%%%%%%%%%%%%%%%%%
%% Title (please edit and customize):                            %%
%%%%%%%%%%%%%%%%%%%%%%%%%%%%%%%%%%%%%%%%%%%%%%%%%%%%%%%%%%%%%%%%%%%
\runtitle{Introduction to the \jphilart Class}
\title{%
    Introduction to the \jphilart Class%
    \thanks{Supported by the Journal of Philosophy, Inc.}%
    \thanks{%
        Current maintainer of class file is
        {VTeX, Lithuania} (\texttt{http://www.vtex.lt}).
        Please send all queries to
        \texttt{latex-support@vtex.lt}.
        }%
    } % \thanks is optional
% \subtitle{}
%%%%%%%%%%%%%%%%%%%%%%%%%%%%%%%%%%%%%%%%%%%%%%%%%%%%%%%%%%%%%%%%%%%
%% Authors (please edit and customize):                          %%
%%%%%%%%%%%%%%%%%%%%%%%%%%%%%%%%%%%%%%%%%%%%%%%%%%%%%%%%%%%%%%%%%%%
\begin{aug}
\author{First-Name Surname}
\address{Address of the Author}
\end{aug}
%%%%%%%%%%%%%%%%%%%%%%%%%%%%%%%%%%%%%%%%%%%%%%%%%%%%%%%%%%%%%%%%%%%
%% Review essays and book reviews (please edit and customize):
%%
%%%%%%%%%%%%%%%%%%%%%%%%%%%%%%%%%%%%%%%%%%%%%%%%%%%%%%%%%%%%%%%%%%%
% \begin{review}
% \title{Book title} \author{Book author}.
% Publisher, year. Pages. Price.
% \end{review}
\end{frontmatter}

%%%%%%%%%%%%%%%%%%%%%%%%%%%%%%%%%%%%%%%%%%%%%%%%%%%%%%%%%%%%%%%%%%%
%% Please replace what follows by the body of your article       %%
%%%%%%%%%%%%%%%%%%%%%%%%%%%%%%%%%%%%%%%%%%%%%%%%%%%%%%%%%%%%%%%%%%%

\lettrine{T}he $\LaTeXe$ class \jphilart is designed for typesetting of articles
for the \textit{Journal of Philosophy}. The \jphilart class comes with
a commented sample file called \texttt{sample.tex}. You are probably
reading the PDF version of this sample file, compiled with a pdflatex
engine.

{Papers using the \LaTeX\ class \jphilart are quicker published}.
The \textit{Journal of Philosophy} need a standardized layout for all
papers. For that reason, you are strongly encouraged to use the \LaTeX\
class \jphilart for your papers.

{An easy way to prepare an article for publication in the
    \textit{Journal of Philosophy} is to edit the source file
    \texttt{sample.tex} for this document. Replace the main
    body of the file with the main body of your article. Supply
    all metadata (title, authors, affiliations) that are requested
    in the latex file.}

You need a copy of the \texttt{jphilart.cls} file in your directory%
\footnote{Or in any location scanned for \texttt{cls} files by your
latex engine.} in order to compile documents based on the \jphilart
class, such as \texttt{sample.tex}.

The \jphilart document class loads automatically the following packages:
\begin{center}
  \ttfamily
  amsmath, amsthm, amsfonts, amssymb, graphicx, enumitem, fontenc,
  baskervald, lettrine, textcase
\end{center}
It is thus not necessary to add \verb+\usepackage+ load commands for
these packages to your latex file. However, you may want to load
additional packages, such as the \emph{cleveref} package by using a
\verb+\usepackage+ command. The precise location of these extra load
commands is clearly mentioned in the \texttt{sample.tex} file. The
\jphilart class provides various environments, and also important commands
such as \verb+\title+, \verb+\author+, etc.

\section{Fonts}

The default font used by the \jphilart class is \emph{NewBaskerville},
i.e. it is checked if \texttt{t1pnb.fd} file exists in \LaTeX\ system.
If it's not \emph{Baskervaldx} package is used as alternative.

\emph{Latin Modern Sans} and \emph{Latin Modern Mono} is used for
\textsf{sans serif} and \texttt{typewriter} fonts.\vfill\eject

New font shape macro \verb+\textscit+ is introduced for \emph{italic}+\textsc{small caps} shape.

The first letter of the article is typeset larger than the rest of the text.
\jphilart class loads the \texttt{lettrine} package for this purpose.

\section{Section headings}

In \jphilart class section titles starting with \verb+\subsection+ are formatted as a
run-in title, like the standard \verb+\paragraph+. Also at the end of these
titles punctuation will be automatically inserted if missing.

\subsection{A sub-section without punctuation} Some text

\subsubsection{Another sub-sub-section with punctuation.} Some text

\paragraph{A paragraph with exclamation mark!} Some text

\subparagraph{A sub-paragraph with question mark?} Some text

\section{Equations}\label{se:mysection}

\jphilart class loads an \texttt{amsmath} package for mathematical typesseting.
Display equations and their numbers are placed on the left-hand side.

The following numbered displayed equation is the first in section \ref{se:mysection}:%
\begin{equation}\label{eq:myequation}
d(\alpha) \in \mathcal{D};
\end{equation}
It is produced with the following source code:
\begin{verbatim}
\begin{equation}\label{eq:myequation}
d(\alpha) \in \mathcal{D};
\end{equation}\end{verbatim}
You may refer to it by using \verb+\eqref{eq:myequation}+ which
produces~\eqref{eq:myequation}.

Here is another numbered displayed equation
\begin{equation}
\begin{split}
&d(\neg\phi) = d(\phi);\\
&d(\phi\wedge\psi) = d(\phi)\odot d(\psi);\\
&d(\phi\vee\psi) = d(\phi)\odot d(\psi);
\end{split}
\end{equation}

\section{Quotations}

You may add a quotation by using the \texttt{quote} environment.
\begin{quote}
   This is the body of the quotation.
   It is inserted with the following environment
   \begin{verbatim}
       \begin{quote}
           This is the body of ...
       \end{quote}\end{verbatim}
\end{quote}

\section{Enunciations}

\jphilart class loads \texttt{asmthm} package for typesetting enunciation
environments. Author may use \verb+\newtheorem+ command in order to define
environment for further use. The precise location of these extra load
commands is clearly mentioned in the \texttt{sample.tex} file.

\begin{theorem}[My theorem]\label{th:1}
    This is the body of the theorem. This theorem has a name between
    parentheses, and this is implemented by adding an optional parameter
    between square brackets to the theorem environment, namely
   \begin{verbatim}
       \begin{theorem}[My theorem] \label{th:1}
           This is the body of ...
       \end{theorem}\end{verbatim}
\end{theorem}

\begin{proof}[Proof of Theorem \ref{th:1}]
    This is the body of the proof of the theorem above. This proof has a name,
    and this is implemented by adding an optional parameter between square
    brackets to the proof environment, namely
    \begin{verbatim}
        \begin{proof}[Proof of Theorem \ref{th:1}]
            This is the body of the proof of ...
        \end{proof}\end{verbatim}
    The proof ends at the square box.
\end{proof}

Note that a square box $\square$ is automatically added at the end of the
proof by the environment ``proof''.

Let us give some more examples of environments in action.

\begin{fact*}[My fact]
    Body of the Fact without a counter.
    It is achieved by using asterisks in the definition, namely
    \begin{verbatim}
        \newtheorem*{fact*}{Fact}\end{verbatim}
\end{fact*}

\begin{proof}
    This is the body of a proof environment without name, obtained using
\end{proof}

\begin{proposition}
    Body of the proposition. This proposition does not have a name.
    Its counter is related to theorem environment counter by the definition
    \begin{verbatim}
        \newtheorem{proposition}[theorem]{Proposition}\end{verbatim}
\end{proposition}

\section{How to include graphics}

You may include graphics as follows
\begin{small}
\begin{verbatim}
\begin{figure}[htbp]
  \centering % gives better spacing than \begin{center}...\end{center}
  \includegraphics[scale=1.0]{filename}
  \caption{This is my figure.}
  \label{fi:myfigure}
\end{figure}\end{verbatim}
\end{small}

Note that in a figure environment, the \verb+\label+ should always appear
after a \verb+\caption+ in order to produce a valid reference to the figure.
You may play with the options \verb+[htbp]+ (see the \LaTeXe\ documentation
for their meaning) and with the options of the \verb+\includegraphics+ command
(see the documentation of the \texttt{graphicx} package).

\section{How to include bibliography}



No bibliographies will be published; bibliographical
material should be put into notes\footnote{Journal citations should include:
the first and last name of author(s), name of journal in full, volume number,
issue number, month or season, year, the article's full range of pages, and,
if applicable, the specified page or pages.

Book references should include: the first and last name of author,
publisher, U.S. city of publication, and date.}.


%%%%%%%%%%%%%%%%%%%%%%%%%%%%%%%%%%%%%%%%%%%%%%%%%%%%%%%%%%%%%%%%%%%
%% You have reached the end of your document.                    %%
%%%%%%%%%%%%%%%%%%%%%%%%%%%%%%%%%%%%%%%%%%%%%%%%%%%%%%%%%%%%%%%%%%%
\end{document}

%% EOF
